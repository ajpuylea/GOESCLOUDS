\documentclass{article}
\usepackage[utf8]{inputenc}

\title{Reconciling backlogged Cloud Cover Data over the Pierre Auger Observatory
Using GOES Satellite Data}
\author{Brain Fick, Andrew Puyleart}
\date{September 2018}

\begin{document}

\maketitle

\section{Introduction}

The Pierre Auger Observatory is a cosmic ray observatory located near Malargue
Argentina. It has two main detection methods a surface detector (SD) and a
fluorescence detector (FD) \cite{P_Abreu}. The fluorescence detectors operate
during the night hours, detecting the UV light emitted from secondary particles
interacting with the nitrogen in the atmosphere. This paper will focus entirely
on the FDs. A cloud detection method was developed previous that used infrared
capabilities of the GOES class satellites [2]. However it has been some time
since the published database has been updated. For collaborators to verify that
there work is valid an update is necessary. Having a robust cloud detection
method is imperative for the integrity of any science done with the FD because
clouds can obscure the UV light emissions from reaching the FD detectors. The
images that the FDs produce when cloud cover is present can mimic interesting
cosmic ray shower events. An effort was made to make the database modern and
updated. All  backed up data has been processed and updated through December
2017. During the month of December 2017 the GOES-13 satellite has been put into
retirement and the new GOES-16 satellite has taken over its duties. The old
method may need some revision to process the new GOES-16 image data. We are
confident that the update using the old methods is correct. We independently
verified using two methods that the new data being made available is of good
quality. The first method used GOES-13’s visible waveband to check against our
infrared method. We also used a data intense method that checked a pixels
yearly normal color distribution versus the total distribution of a years’
worth of data for all pixels.  

\section{Visible light method}

Although the FD detectors operate at night and the IR method of cloud detection
is not normally used during day time hours we thought it would be worth seeing
if our method closely resembled what is seen during the day. We used GOES-13
visible light capabilities and our infrared algorithm and compared them side by
side. 

(insert image) 

Figure 1. On the left: A visible light spectrum image from the GOES-13
satellite. On the right: An image made using our IR algorithm of the exact same
time. From inspection you can see pretty good agreement despite the visible
image being 4 times the resolution. 

Based on these images we concluded that our IR-method was successfully
processing day-time cloud cover. Some adjustment had to be made in the
algorithm for the IR-images to work during the day. However We don’t recommend
the use of this IR method during the day. Several dozens of the images were
produced as seen in figure 1. 

 \section{Pixel Color Distribution} 

 To determine if the images had any sort of pattern or if the cloud
distribution across the images were truly random a pixel analysis was done. The
first step was to determine the average number of occurrences of a given color
the pixel could be. Each pixel has up to 5 colors corresponding to how probable
a cloud is in the pixel. We chose to use a grey scale for displaying the
probability chances \ref{fig:CloudMap}. Once the average was found for each
color a normal distribution was assumed and a standard deviation of number of
occurrences was calculated. 

\begin{figure}{hbp} 
\includegraphics[width = 0.85\textwidth]{CldMap.png}
 \caption{A sample cloud probability map on the left. A grey scale is used to
display how likely a cloud is occupying the area in the pixel. The Auger array
outline is seen in red. The FD are the colored fans.} 
 \label{fig:CloudMap}
\end{figure}

Pixels behaving poorly were identified as having above or below a two sigma
count number from the average. These pixels were plotted on a map of the
observatory. After seeing some coincidental pixels behaving poorly through
consecutive years of cloud data we decided to investigate the topography of the
area \ref{fig:2016_2017}. 

\begin{figure}{hbp} 
\includegraphics[width = 0.85\textwidth]{BadPixLocations2016.png, BadPixLocations2017.png} 
  \caption{On the left a is a graph of all two and three sigma count pixels for the year
2016. On the right is 2017’s two and three sigma count pixels. You can see some
reoccurring patterns between the years like the feature from pixels 15-19 and
pixel 359.  Pixels with counts greater than 2 sigma but less than 3 are seen in
yellow. Pixels with counts greater than 3 sigma are seen in orange.}
 \label{fig:2016_2017} 
\end{figure}

A new map was generated with the misbehaving pixel array overlayed on a
topographic map. Some features were identified as possible candidates for
elevation causes such as the foothills near Malargue, and the line of inactive
volcanoes that appear on the bottom right of the area surveyed
\ref{fig:Elevation09}. 

\begin{figure}{hbp}
\includegraphics[width = 0.85\textwidth]{BadPixLocations2009,
BadPixElevations09.png} 
  \caption{On the left:  a graph showing each pixels ID
number. The pixels that have a two sigma count number are filled in yellow, the
three sigma in orange. The Auger array is outlined in salmon. On the right is
the same data plotted on an elevation map. Red and purple were chosen for the
two and three sigma count numbers for easier reading. You can see some
elevation features that may be the cause of these larger problems like on the
bottom right the old volcano and the bottom left the mountain foothills.}
 \label{fig:Elevation09}
\end{figure}

However not all geographical reasons for misbehaving pixels could be claimed. A
program was written that determined how likely two numerically sequential
pixels could have 2 or 3 sigma deviations from the average. It was found that
for a given year there was between an 80-90 percent chance that at least one 2
sigma or 3 sigma pixel’s neighbor could also be misbehaving. Considering that
each of our graphs contains 5 different distributions (one for each color the
pixel can be) the likelihood of two or more pixels having behavior that
deviated from the average is nearly guaranteed. So the “clumping” behaviors we
see in these graphs could be considered random and not an artifact of our
algorithm behaving inappropriately. 
  
\section{Conclusion}
Taking everything into consideration we are happy to give our blessing to the
last of GOES-13’s data and have updated the cloud database for our
collaborators to use. The work that remains is transitioning the cloud
detection method  into a new phase of its life working with the GOES-16
satellite. We are excited to see how it improves the spacial resolution of our
IR observation technique. 
\section{Citations}
\bibitem{P_Abreu,
 author = P. Abreu, et al,
 title = ,
 journal =,
 year = 2013,
 volume = ,
 pages = ,
}
 
\end{document}
